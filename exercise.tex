\documentclass{article}

\usepackage{amsfonts}
\usepackage{bussproofs}
\usepackage[margin=1.5in]{geometry}
\usepackage{hyperref}
\usepackage{listings}
\usepackage{xcolor}

\title{
    Exercises Submission
    \\ \Large
    Automated Analysis of Security Protocols
    }
\author{Elia Scandaletti - 2087934}
\date{June 19, 2023}

\newcommand{\pk}[1]{\texttt{pk}(#1)}
\newcommand{\senc}[2]{\texttt{senc}(#1, #2)}
\newcommand{\aenc}[2]{\texttt{aenc}(#1, #2)}
\newcommand{\pair}[2]{\left\langle #1, #2\right\rangle}
\newcommand{\sdec}[2]{\texttt{sdec}(#1, #2)}
\newcommand{\adec}[2]{\texttt{adec}(#1, #2)}
\newcommand{\Sdec}[2]{\sdec{\senc{#1}{#2}}{#2}}
\newcommand{\Adec}[2]{\adec{\aenc{#1}{\pk{#2}}}{#2}}
\newcommand{\fst}[1]{\texttt{fst}(#1)}
\newcommand{\snd}[1]{\texttt{snd}(#1)}
\newcommand{\out}[2]{\texttt{out}(#1, #2)}
\newcommand{\inn}[2]{\texttt{inn}(#1, #2)}

\lstdefinelanguage{apip}{
    morekeywords=[1]{
	    channel, type, fun, reduc, forall, event, let, in, out, new, if, then, 0, query, process
    },
    sensitive=true,
    morecomment=[s]{(*}{*)}
}

\definecolor{codegreen}{rgb}{0,0.6,0}
\definecolor{codegray}{rgb}{0.5,0.5,0.5}
\definecolor{codepurple}{rgb}{0.58,0,0.82}
\definecolor{backcolour}{rgb}{0.95,0.95,0.92}

\lstdefinestyle{apip}{
    language={apip},
	stringstyle=\rm\textup,
    backgroundcolor=\color{backcolour},
    commentstyle=\color{codegreen},
    keywordstyle=\color{magenta},
    numberstyle=\tiny\color{codegray},
    stringstyle=\color{codepurple},
    basicstyle=\ttfamily,
    breakatwhitespace=false,
    breaklines=true,
    captionpos=b,
    keepspaces=true,
    numbers=left,
    numbersep=5pt,
    showspaces=false,
    showstringspaces=false,
    showtabs=false,
    tabsize=2
}
\lstset{style=apip}

\begin{document}
\maketitle

\paragraph{Exercise 1}
\subparagraph{1.}
The following tree solves the deduction
$$
    sk_A,
    sk_B,
    \aenc{n_A}{\pk{sk_B}},
    \senc{\aenc{n_B}{\pk{sk_A}}}{n_A},
    \senc{s}{\pair{n_A}{n_B}}
    \vdash_{\mathcal{I}_{DY}} s
$$

\begin{center}
    \makebox[0pt][c]{%
        \begin{minipage}{\paperwidth}
            \begin{prooftree}
                \AxiomC{$\senc{s}{\pair{n_A}{n_B}}$}
                \AxiomC{$\aenc{n_A}{\pk{sk_B}}$}
                \AxiomC{$sk_B$}
                \BinaryInfC{$n_A$}
                \AxiomC{$\senc{\aenc{n_B}{\pk{sk_A}}}{n_A}$}
                \AxiomC{$\aenc{n_A}{\pk{sk_B}}$}
                \AxiomC{$sk_B$}
                \BinaryInfC{$n_A$}
                \BinaryInfC{$\aenc{n_B}{\pk{sk_A}}$}
                \AxiomC{$sk_A$}
                \BinaryInfC{$n_B$}
                \BinaryInfC{$\pair{n_A}{n_B}$}
                \BinaryInfC{$s$}
            \end{prooftree}
        \end{minipage}
    }
\end{center}

\subparagraph{2.}
$\mathcal{I}_{DY}$ is called a local theory because, by definition of local theory, for any finite set of terms $S$, for any term $t$ such that $S \vdash_{\mathcal{I}_{DY}} t$, there exist a proof tree $\Pi$ of $S \vdash_{\mathcal{I}_{DY}} t$ such that every label in $\Pi$ is in $\texttt{st}(S) \cup \{t\}$.

This property of the $\mathcal{I}_{DY}$ inference system can be proven by induction on the size of $\Pi$.

$\mathcal{I}_{DY}$ being a local theory implies that for any $S$, for any $t$, $S \vdash_{\mathcal{I}_{DY}} t$ is decidable in polynomial time.

\paragraph{Exercise 2}
\subparagraph{1.}
The following tree solves the deduction
$$
    sk_A,
    sk_B,
    \aenc{n_A}{\pk{sk_B}},
    \senc{\aenc{n_B}{\pk{sk_A}}}{n_A},
    \senc{s}{\pair{n_A}{n_B}}
    \vdash_{E_{enc}} s
$$

\begin{center}
    \makebox[0pt][c]{%
        \begin{minipage}{\paperwidth}
            \begin{prooftree}
                \AxiomC{$\senc{s}{\pair{n_A}{n_B}}$}
                \AxiomC{$\aenc{n_A}{\pk{sk_B}}$}
                \AxiomC{$sk_B$}
                \BinaryInfC{$\Adec{n_A}{sk_B}$}
                \UnaryInfC{$n_A$}
                \AxiomC{$\senc{\aenc{n_B}{\pk{sk_A}}}{n_A}$}
                \AxiomC{$\aenc{n_A}{\pk{sk_B}}$}
                \AxiomC{$sk_B$}
                \BinaryInfC{$\Adec{n_A}{sk_B}$}
                \UnaryInfC{$n_A$}
                \BinaryInfC{$\Sdec{\aenc{n_B}{\pk{sk_A}}}{n_A}$}
                \UnaryInfC{$\aenc{n_B}{\pk{sk_A}}$}
                \AxiomC{$sk_A$}
                \BinaryInfC{$\Adec{n_B}{sk_A}$}
                \UnaryInfC{$n_B$}
                \BinaryInfC{$\pair{n_A}{n_B}$}
                \BinaryInfC{$\Sdec{s}{\pair{n_A}{n_B}}$}
                \UnaryInfC{$s$}
            \end{prooftree}
        \end{minipage}
    }
\end{center}

\paragraph{Exercise 3}
\subparagraph{1.}
Let $M = \texttt{h}(y)$ and $N = x$.
Then we have $(M \neq_{E_{enc}} N)_{\varphi_1} \mbox{ and } (M =_{E_{enc}} N)_{\varphi_2}$.

\subparagraph{2.}
$\varphi_1$ and $\varphi_2$ are statically equivalent because their only ``difference'' is hidden behind an asymmetric encryption of which the private key is not known and therefore cannot be accessed by decrypting $x$.
Additionally, it is not possible to create a term equivalent to $x$ because it would require knowing $k$.

\paragraph{Exercise 4}
\subparagraph{1.}
A witness of $A \not\approx B$ is the following context
$$
    C[\_] =
    \out{c}{0}.
    \out{c}{1}.
    \inn{c}{pk_k}.
    \inn{c}{t}.
    \texttt{if } t \neq \aenc{\pair{0}{1}}{pk_k}
    \texttt{ then } \out{a}{0}
    \ ||\ \_
$$

We have $C[A] \rightarrow^* \out{a}{0}$, while $C[B]$ will never emit on $a$.

\paragraph{Exercise 5}

The following listing is the code used to represent the given protocol.

The first query encode the authentication property of the protocol which is verified.
It can be read as ``If $B$ receives a message $m$ and thinks it is from $A$, then $A$ has sent that message''.

The second properties is integrity and can be read as ``If $A$ acknowledge that $B$ has correctly received the message $m$, then $B$ has correctly received the message $m$ from $A$''.
This property is not verified, since ProVerif assumes the attacker knows the nonce $N_b$ generated by $B$ and can therefore intercept the message $\{m\}_{N_b}$, decode $m$ and respond to $A$ with $\texttt{h}(m)$.

The last property is confidentiality and states that $A$ and $B$ are the only ones that know $m$.
It can be written as ``It is never the case that the attacker knows $m$ and $A$ acknowledges that $m$ has been correctly received by $B$''.
ProVerif stated that this property does not hold, assuming -as for the previous property- that the attacker knows $N_b$.

The verification has been performed using the online demo of ProVerif at \url{http://proverif20.paris.inria.fr/}.

\lstinputlisting{protocol.pv}

\end{document}