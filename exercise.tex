\documentclass{article}

\usepackage{amsfonts}
\usepackage{bussproofs}
\usepackage[margin=1.5in]{geometry}
\usepackage{lscape}

\title{
    Exercises Submission
    \\ \Large
    Automated Analysis of Security Protocols
    }
\author{Elia Scandaletti - 2087934}

\newcommand{\pk}[1]{\texttt{pk}(#1)}
\newcommand{\senc}[2]{\texttt{senc}(#1, #2)}
\newcommand{\aenc}[2]{\texttt{aenc}(#1, #2)}
\newcommand{\pair}[2]{\left\langle #1, #2\right\rangle}
\newcommand{\sdec}[2]{\texttt{sdec}(#1, #2)}
\newcommand{\adec}[2]{\texttt{adec}(#1, #2)}
\newcommand{\Sdec}[2]{\sdec{\senc{#1}{#2}}{#2}}
\newcommand{\Adec}[2]{\adec{\aenc{#1}{\pk{#2}}}{#2}}
\newcommand{\fst}[1]{\texttt{fst}(#1)}
\newcommand{\snd}[1]{\texttt{snd}(#1)}

\begin{document}
\maketitle

\paragraph{Exercise 1}
\subparagraph{1.}
The following tree solves the deduction
$$
    sk_A,
    sk_B,
    \aenc{n_A}{\pk{sk_B}},
    \senc{\aenc{n_B}{\pk{sk_A}}}{n_A},
    \senc{s}{\pair{n_A}{n_B}}
    \vdash_{\mathcal{I}_{DY}} s
$$

\begin{center}
    \makebox[0pt][c]{%
        \begin{minipage}{\paperwidth}
            \begin{prooftree}
                \AxiomC{$\senc{s}{\pair{n_A}{n_B}}$}
                \AxiomC{$\aenc{n_A}{\pk{sk_B}}$}
                \AxiomC{$sk_B$}
                \BinaryInfC{$n_A$}
                \AxiomC{$\senc{\aenc{n_B}{\pk{sk_A}}}{n_A}$}
                \AxiomC{$\aenc{n_A}{\pk{sk_B}}$}
                \AxiomC{$sk_B$}
                \BinaryInfC{$n_A$}
                \BinaryInfC{$\aenc{n_B}{\pk{sk_A}}$}
                \AxiomC{$sk_A$}
                \BinaryInfC{$n_B$}
                \BinaryInfC{$\pair{n_A}{n_B}$}
                \BinaryInfC{$s$}
            \end{prooftree}
        \end{minipage}
    }
\end{center}

\subparagraph{2.}
$\mathcal{I}_{DY}$ is called a local theory because, by definition of local theory, for any finite set of terms $S$, for any term $t$ such that $S \vdash_{\mathcal{I}_{DY}} t$, there exist a proof tree $\Pi$ of $S \vdash_{\mathcal{I}_{DY}} t$ such that every label in $\Pi$ is in $\texttt{st}(S) \cup \{t\}$.

This property of the $\mathcal{I}_{DY}$ inference system can be proven by induction on the size of $\Pi$.

$\mathcal{I}_{DY}$ being a local theory implies that for any $S$, for any $t$, $S \vdash_{\mathcal{I}_{DY}} t$ is decidable in polynomial time.

\paragraph{Exercise 2}
\subparagraph{1.}
The following tree solves the deduction
$$
    sk_A,
    sk_B,
    \aenc{n_A}{\pk{sk_B}},
    \senc{\aenc{n_B}{\pk{sk_A}}}{n_A},
    \senc{s}{\pair{n_A}{n_B}}
    \vdash_{E_{enc}} s
$$

\begin{center}
    \makebox[0pt][c]{%
        \begin{minipage}{\paperwidth}
            \begin{prooftree}
                \AxiomC{$\senc{s}{\pair{n_A}{n_B}}$}
                \AxiomC{$\aenc{n_A}{\pk{sk_B}}$}
                \AxiomC{$sk_B$}
                \BinaryInfC{$\Adec{n_A}{sk_B}$}
                \UnaryInfC{$n_A$}
                \AxiomC{$\senc{\aenc{n_B}{\pk{sk_A}}}{n_A}$}
                \AxiomC{$\aenc{n_A}{\pk{sk_B}}$}
                \AxiomC{$sk_B$}
                \BinaryInfC{$\Adec{n_A}{sk_B}$}
                \UnaryInfC{$n_A$}
                \BinaryInfC{$\Sdec{\aenc{n_B}{\pk{sk_A}}}{n_A}$}
                \UnaryInfC{$\aenc{n_B}{\pk{sk_A}}$}
                \AxiomC{$sk_A$}
                \BinaryInfC{$\Adec{n_B}{sk_A}$}
                \UnaryInfC{$n_B$}
                \BinaryInfC{$\pair{n_A}{n_B}$}
                \BinaryInfC{$\Sdec{s}{\pair{n_A}{n_B}}$}
                \UnaryInfC{$s$}
            \end{prooftree}
        \end{minipage}
    }
\end{center}

\paragraph{Exercise 3}
\subparagraph{1.}
Let $M = \texttt{h}(y)$ and $N = x$.
Then we have $(M \neq_{E_{enc}} N)_{\varphi_1} \mbox{ and } (M =_{E_{enc}} N)_{\varphi_2}$.

\subparagraph{2.}
$\varphi_1$ and $\varphi_2$ are statically equivalent because their only ``difference'' is hidden behind an asymmetric encryption of which the private key is not known and therefore cannot be accessed by decrypting $x$.
Additionally, it is not possible to create a term equivalent to $x$ because it would require knowing $k$.

\end{document}